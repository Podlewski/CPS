\documentclass[12pt]{article}
\usepackage[T1]{fontenc}
\usepackage[T1]{polski}
\usepackage[cp1250]{inputenc}
\newcommand{\BibTeX}{{\sc Bib}\TeX} 
\usepackage{graphicx}
\usepackage{amsfonts}
\usepackage{float}

\setlength{\textheight}{21cm}

\title{{\bf Zadanie nr 2 - Pr�bkowanie i kwantyzacja}\linebreak
Cyfrowe Przetwarzanie Sygna��w}
\author{Justyna Hubert, 210200 \and Karol Podlewski, 210294}
\date{17.04.2019}

\begin{document}
\clearpage\maketitle
\thispagestyle{empty}
\newpage
\setcounter{page}{1}


%%%%%%%%%%%%%%%%%%%%%%%%%%%%%%%%%%%%%%%%%%%%%%%%%%%%%%%%%%%%
%% CEL ZADANIA
%%%%%%%%%%%%%%%%%%%%%%%%%%%%%%%%%%%%%%%%%%%%%%%%%%%%%%%%%%%%
\section{Cel zadania}
Celem �wiczenia jest zapoznanie si� z praktycznymi aspektami procesu konwersji
analogowo-cyfrowej (A/C) i cyfrowo-analogowej (C/A) sygna��w. 

%%%%%%%%%%%%%%%%%%%%%%%%%%%%%%%%%%%%%%%%%%%%%%%%%%%%%%%%%%%%
%% WST�P TEORETYCZNY
%%%%%%%%%%%%%%%%%%%%%%%%%%%%%%%%%%%%%%%%%%%%%%%%%%%%%%%%%%%%
\section{Wst�p teoretyczny}
Zadanie polega�o na zaimplementowaniu procesu przetwarzania analogowo-cyfrowego z uwzgl�dnieniem operacji pr�bkowania i kwantyzacji oraz konwersji odwrotnej, tj. cyfrowo-analogowej. Zosta�y wykonane nast�puj�ce warianty:
\begin{itemize}
	\item (S1) Pr�bkowanie r�wnomierne,
	\item (Q1) Kwantyzacja r�wnomierna z obci�ciem,
	\item (R2) Interpolacja pierwszego rz�du,
	\item (R3) Rekonstrukcja w oparciu o funkcj� sinc.
\end{itemize}
Ponadto, w ramach realizacji �wiczenia nale�a�o zaimplementowa� nast�puj�ce miary:
\begin{itemize}
	\item B��d �redniokwadratowy (MSE),
	\item Stosunek sygna� - szum (SNR),
	\item Szczytowy stosunek sygna� - szum (PSNR).
	\item Maksymalna r�nica (MD),
	\item Efektywna liczba bit�w (ENOB).
\end{itemize}
\pagebreak

%%%%%%%%%%%%%%%%%%%%%%%%%%%%%%%%%%%%%%%%%%%%%%%%%%%%%%%%%%%%
%% SEKCJA EKSPERYMENT�W
%%%%%%%%%%%%%%%%%%%%%%%%%%%%%%%%%%%%%%%%%%%%%%%%%%%%%%%%%%%%
\section{Eksperymenty i wyniki}

%%%%%%%%%%%%%%%%%%%%%%%%%%%%%%%%%%%%%%%%%%%%%%%%%%%%%%%%%%%%
%%%%%%%%%%%% EKSPERYMENT #1  
\subsection{Sygna� sinusoidalny}
\pagebreak

%%%%%%%%%%%%%%%%%%%%%%%%%%%%%%%%%%%%%%%%%%%%%%%%%%%%%%%%%%%%
%%%%%%%%%%%% EKSPERYMENT #2  
\subsection{Sygna� prostok�tny}
\pagebreak

%%%%%%%%%%%%%%%%%%%%%%%%%%%%%%%%%%%%%%%%%%%%%%%%%%%%%%%%%%%%
%%%%%%%%%%%% EKSPERYMENT #3  
\subsection{Sygna� tr�jk�tny}
\pagebreak

%%%%%%%%%%%%%%%%%%%%%%%%%%%%%%%%%%%%%%%%%%%%%%%%%%%%%%%%%%%%
%% WNIOSKI
%%%%%%%%%%%%%%%%%%%%%%%%%%%%%%%%%%%%%%%%%%%%%%%%%%%%%%%%%%%%
\section{Wnioski}
Aplikacja zosta�a napisana zgodnie z instrukcj� do zadania [1]. Program
poprawnie implementuje konwersj� A/C oraz C/A wraz z obliczaniem parametr�w. Aplikacja zosta�a napisana w spos�b, aby umo�liwiaj�cy nam rozszerzenie jej o kolejne funkcjonalno�ci.

%%%%%%%%%%%%%%%%%%%%%%%%%%%%%%%%%%%%%%%%%%%%%%%%%%%%%%%%%%%%
%% BIBLIOGRAFIA
%%%%%%%%%%%%%%%%%%%%%%%%%%%%%%%%%%%%%%%%%%%%%%%%%%%%%%%%%%%%
\renewcommand\refname{Bibliografia}
\bibliographystyle{plain}
\bibliography{Zad2_Bibliografia}

\end{document}
